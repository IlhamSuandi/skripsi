\textbf{\section{Sumber Data}}
Sumber data yang digunakan dalam penelitian ini adalah data saham indeks yang tergolong utama yaitu LQ45 dengan rentang waktu 5-10 tahun terakhir yang akan diperoleh dari \href{https://finance.yahoo.com}{yahoo finance}. Data yang akan diperoleh meliputi:
\begin{itemize}
	\item \textit{High}: Harga saham tertinggi
	\item \textit{Low}: Harga saham terendah
	\item \textit{Close}: Harga saham terakhir/penutupan
	\item \textit{Volume}: Volume saham
\end{itemize}

Data ini akan diunduh dalam format \textit{CSV} untuk mempermudah proses analisis menggunakan perangkat lunak seperti Python dengan bantuan library seperti yfinance. Selain Yahoo Finance, jika diperlukan, data tambahan dapat diperoleh dari sumber alternatif seperti laporan tahunan perusahaan terkait atau data dari Bursa Efek Indonesia (BEI).

