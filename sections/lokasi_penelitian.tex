\textbf{\section{Lokasi Penelitian}}
\indent

Dikarenakan data yang akan digunakkan dalam penelitian ini dapat diperoleh dari melalui internet, maka penelitian ini dapat dilakukan dimana saja atau secara daring. Dalam penelitian ini, peneliti menggunakan data yang bersumber pada \href{https://finance.yahoo.com}{yahoo finance}. Alasan menggunakan yahoo finance? karena yahoo finance menyediakan data historis saham yang sangatlah lengkap bukan hanya saham Indonesia saja, melainkan saham international juga. Selain itu data yang disediakan oleh yahoo finance memiliki isi yang sangat relevan bahkan mencakup \textit{OHLC (Open, High, Low, Close)}, volume, serta indikator teknikal lainnya.

\begin{figure}[H]
	\centering
	\includegraphics[width=0.5\textwidth]{yahoo_finance.png}
	\caption{Logo Yahoo Finance}
\end{figure}

% TODO: lengkapi lagi
Yahoo Finance adalah platform yang sering digunakan untuk data keuangan dan memiliki banyak keunggulan yang membuatnya pilihan ideal untuk analisis harga saham, terutama ketika menggunakan data historis. Berikut alasan mengapa Yahoo Finance sangatlah digunakan dalam penelitian ini:
\begin{itemize}
  	\item \textit{Gratis}: Yahoo Finance menyediakan data yang gratis dan dapat digunakan untuk penelitian tanpa harus membayar biaya.
	\item \textit{Data Historis}: Yahoo Finance menyediakan data historis saham yang sangatlah lengkap dan dapat digunakan untuk analisis harga saham. Data ini dapat dikumpulkan dari berbagai sumber seperti Yahoo Finance, Yahoo Finance Indonesia, dan Yahoo Finance Global.
	\item \textit{Relevansi}: Data yang disediakan oleh Yahoo Finance memiliki isi yang sangat relevan dan dapat digunakan untuk analisis harga saham. Data ini juga memiliki indikator teknikal yang dapat digunakan untuk menganalisis data saham.
	\item \textit{Integrasi dengan python}: Yahoo Finance dapat diakses menggunakan \textit{python} dengan menggunakan \textit{library} yaitu \textit{`yfinance`}. Dengan hal ini penelitian ini akan dapat mudah dalam melakukan proses preprocessing data sebelum digunakan pada model machine learning seperti LSTM.
\end{itemize}
