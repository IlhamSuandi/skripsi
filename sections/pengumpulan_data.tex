\textbf{\section{Teknik Pengumpulan Data}}

% TODO: tambahkan lagi
Pengumpulan data dalam penelitian ini bertujuan untuk mendapatkan data historis saham yang terdaftar dalam indeks LQ45 di Bursa Efek Indonesia (BEI). Indeks LQ45 dipilih karena terdiri dari 45 saham yang memiliki likuiditas tinggi dan kapitalisasi pasar besar, sehingga representatif untuk mencerminkan dinamika pasar saham di Indonesia. Proses pengumpulan data dilakukan dengan memanfaatkan library Python bernama yfinance, yang memungkinkan akses langsung ke data historis saham dari Yahoo Finance. Langkah pertama dalam pengumpulan data adalah mengidentifikasi daftar saham yang saat ini masuk dalam indeks LQ45 melalui situs resmi BEI atau sumber terpercaya lainnya. Contoh saham yang termasuk dalam indeks ini adalah BBCA, TLKM, UNVR, ASII, BMRI, dan lainnya.

Selanjutnya, rentang waktu pengumpulan data ditetapkan selama lima tahun terakhir dengan resolusi data harian. Pemilihan rentang waktu ini bertujuan untuk menyediakan data yang cukup panjang untuk menangkap pola tren jangka panjang dan fluktuasi musiman yang mungkin terjadi. Dengan menggunakan yfinance, skrip Python akan dibuat untuk mengunduh data historis setiap saham dalam format yang mencakup informasi harga pembukaan (Open), harga tertinggi (High), harga terendah (Low), harga penutupan (Close), harga penutupan yang disesuaikan (Adj Close), dan volume transaksi (Volume). Data yang diunduh kemudian akan disimpan dalam format CSV agar mudah diakses dan dianalisis.

Setelah data berhasil dikumpulkan, langkah berikutnya adalah melakukan pembersihan data untuk memastikan kualitas dan konsistensinya. Hal ini meliputi penghapusan nilai-nilai kosong (missing values), pengecekan kejanggalan data seperti outlier yang tidak wajar, serta memastikan keseragaman format waktu dan atribut lainnya. Data yang telah dibersihkan akan dinormalisasi menggunakan metode Min-Max Scaling untuk mereduksi skala data ke rentang [0,1]. Normalisasi ini bertujuan untuk meningkatkan performa model LSTM yang sensitif terhadap skala data.

Setelah proses normalisasi, data akan diubah menjadi bentuk time-series sequence agar sesuai dengan kebutuhan model LSTM. Sequence ini dibuat dengan menggunakan beberapa hari data historis sebelumnya sebagai input untuk memprediksi harga saham di hari berikutnya. Sebagai contoh, data selama 10 hari terakhir dapat digunakan untuk memprediksi harga penutupan pada hari ke-11. Proses ini dilakukan untuk setiap saham dalam indeks LQ45, sehingga menghasilkan dataset yang lengkap dan terstruktur.

Data yang telah melalui proses ini akan siap digunakan untuk pelatihan model LSTM. Dengan pendekatan ini, pengumpulan data tidak hanya memastikan ketersediaan informasi yang relevan tetapi juga mempersiapkan dataset yang berkualitas tinggi untuk mendukung proses prediksi harga saham yang akurat. Teknik pengumpulan data menggunakan yfinance ini memberikan efisiensi dan fleksibilitas, sekaligus memastikan bahwa data yang digunakan valid, up-to-date, dan representatif terhadap kondisi pasar saham yang sebenarnya.

