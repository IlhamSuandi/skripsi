\textbf{\section{Teknik Pengumpulan Data}}

% TODO: tambahkan lagi
Teknik penelitian yang akan digunakan dalam penelitian ini adalah studi literatur, yang berfokus pada pengumpulan dan analisis sumber-sumber sekunder yang relevan untuk mendalami topik analisis pergerakan harga saham menggunakan algoritma Long Short-Term Memory (LSTM). Studi literatur ini bertujuan untuk mendapatkan pemahaman yang lebih dalam tentang berbagai konsep yang terkait dengan penelitian, termasuk teori dasar mengenai saham, metode analisis saham seperti analisis fundamental dan teknikal, serta penerapan algoritma deep learning dalam prediksi harga saham.

Proses pertama dalam studi literatur adalah mengidentifikasi dan mengumpulkan berbagai referensi yang relevan, baik berupa buku, artikel ilmiah, jurnal, laporan riset, dan publikasi terkait lainnya yang membahas tentang analisis saham, kecerdasan buatan, serta aplikasi LSTM dalam data keuangan. Sumber-sumber ini akan digunakan untuk menggali teori dasar dan konsep-konsep yang berkaitan dengan analisis teknikal dan prediksi harga saham, serta memberikan gambaran tentang penggunaan algoritma LSTM dalam konteks pasar saham.

Setelah pengumpulan sumber, langkah selanjutnya adalah melakukan analisis terhadap literatur yang ada untuk mengidentifikasi temuan-temuan penting dan tren yang muncul dalam penelitian sebelumnya. Fokus utama adalah untuk membandingkan berbagai pendekatan yang telah digunakan dalam analisis saham, baik yang berbasis analisis fundamental, teknikal, maupun yang menggunakan teknologi kecerdasan buatan seperti deep learning dan LSTM. Selain itu, penelitian ini juga akan mengidentifikasi kelebihan dan kekurangan dari berbagai metode yang ada, serta melihat sejauh mana penggunaan LSTM dapat meningkatkan akurasi prediksi harga saham dibandingkan dengan algoritma lain seperti Recurrent Neural Networks (RNN).

Hasil dari studi literatur ini diharapkan dapat memberikan wawasan yang lebih jelas mengenai penerapan LSTM dalam memprediksi pergerakan harga saham di Indonesia, khususnya saham-saham yang terdaftar dalam LQ45. Studi literatur ini juga akan memberikan dasar teoretis untuk penelitian selanjutnya yang akan melibatkan pengumpulan dan analisis data harga saham historis untuk menguji efektivitas LSTM dalam prediksi harga saham.

