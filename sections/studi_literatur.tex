\textbf{\section{Studi Literatur}}

\begin{center}
  \centering
	\begin{longtable}{| m{1cm} | m{3cm}| p{8cm} |}
		\hline
		\multicolumn{1}{|c|}{\textbf{No.}} & \multicolumn{1}{c|}{\textbf{Kriteria}} & \multicolumn{1}{c|}{\textbf{Isi}}                                                                                                                                                                                                                                                                                                                                                                                                             \\

    \hline
		\endfirsthead

    \hline
    \endhead

    \hline
		\endfoot

		\hline
		\endlastfoot

		\multirow[t]{6}{*}{1.}             & Nama Peneliti                          & \cite{pipin2023prediksi}                                                                                                                                                                                                                                                                                                                                                                                                                                                                                                                                                                                                                                                                                                                                                                                                                                                                                                                                                                                                                                                                                                                                                                                                                                                                                                                                                                                                                                                                                                                                                                                                                                                                                                                                                                                                                                                                                                                                                                                                                                                                                                                                                                                                                                                                                                                                                                                                                                                                                                                                                                                                                                           \\         
		\cline{2-3}
		                                   & Judul Penelitian                       & Prediksi Saham Menggunakan Recurrent Neural Network (RNN-LSTM) dengan Optimasi Adaptive Moment Estimation                                                                                                                                                                                                                                                                                                                                     \\
		\cline{2-3}
		                                   & Permasalahan dan Tujuan                & Penelitian ini bertujuan untuk mengembangkan model prediksi harga saham dengan menggunakan algoritma LSTM, dengan tujuan untuk meningkatkan akurasi prediksi harga saham di pasar Indonesia.                                                                                                                                                                                                                                                  \\
		\cline{2-3}
		                                   & Metode                                 & Penelitian ini menggunakan data historis harga saham perusahaan yang terdaftar di Bursa Efek Indonesia (BEI). Model LSTM dikembangkan untuk mengolah data time series dan digunakan untuk memprediksi harga saham di masa depan.                                                                                                                                                                                                              \\
		\cline{2-3}
		                                   & Hasil                                  & Hasil penelitian menunjukkan bahwa model LSTM memiliki akurasi yang lebih baik dibandingkan dengan model RNN dalam memprediksi harga saham, dengan MAPE (Mean Absolute Percentage Error) yang lebih rendah.                                                                                                                                                                                                                                   \\
		\cline{2-3}
		                                   & Kelebihan dan Kekurangan               & Kelebihan: Penelitian ini berhasil menunjukkan bahwa LSTM dapat menghasilkan prediksi yang lebih akurat dibandingkan dengan metode lain, terutama dalam memproses data yang bersifat non-linear dan kompleks. \newline Kekurangan: Penelitian ini hanya membandingkan dua model (LSTM dan RNN) tanpa mempertimbangkan algoritma lain yang mungkin lebih sesuai untuk jenis data tertentu, seperti model berbasis transformator (Transformer). \\

    \hline
		\multirow[t]{6}{*}{2.}             & Nama Peneliti                          & \cite{moghar2020stock} \\
		\cline{2-3}
		                                   & Judul Penelitian                       & Stock Market Prediction Using LSTM Recurrent Neural Network \\
		\cline{2-3}
		                                   & Permasalahan dan Tujuan                & Penelitian ini membahas tantangan dalam memprediksi harga saham, terutama karena volatilitas tinggi dan pola non-linear dalam data. Tujuannya adalah mengevaluasi efektivitas model LSTM dalam memprediksi harga saham berdasarkan data historis. \\
		\cline{2-3}
		                                   & Metode                                 & Menggunakan data historis harga saham, model LSTM diimplementasikan dengan berbagai parameter untuk menangkap pola non-linear. Evaluasi dilakukan terhadap performa model dalam memprediksi harga saham masa depan. \\
		\cline{2-3}
		                                   & Hasil                                  & LSTM memberikan hasil yang lebih akurat dibandingkan model tradisional seperti ARIMA, khususnya dalam menangkap pola volatilitas saham. \\
		\cline{2-3}
		                                   & Kelebihan dan Kekurangan               & Kelebihan: Efektif dalam menangani data time series dengan pola kompleks dan non-linear.\newline Kekurangan: Bergantung pada parameter model yang harus dioptimalkan untuk hasil terbaik. \\

    \hline
		\multirow[t]{6}{*}{3.}             & Nama Peneliti                          & \cite{alim2023pemodelan} \\
		\cline{2-3}
		                                   & Judul Penelitian                       & Pemodelan Time Series Data Saham LQ45 dengan Algoritma LSTM, RNN, dan ARIMA \\
		\cline{2-3}
		                                   & Permasalahan dan Tujuan                & Penelitian ini bertujuan membandingkan performa algoritma LSTM, RNN, dan ARIMA dalam memprediksi harga saham LQ45 untuk menentukan model yang paling akurat. \\
		\cline{2-3}
		                                   & Metode                                 & Dataset saham LQ45 diproses untuk membangun model LSTM, RNN, dan ARIMA. Evaluasi dilakukan menggunakan metrik akurasi seperti MSE (Mean Squared Error) dan MAE (Mean Absolute Error). \\
		\cline{2-3}
		                                   & Hasil                                  & LSTM memberikan hasil terbaik dalam memprediksi harga saham dibandingkan RNN dan ARIMA karena kemampuannya menangkap pola jangka panjang dan non-linear. \\
		\cline{2-3}
		                                   & Kelebihan dan Kekurangan               & Kelebihan: Memberikan perbandingan kinerja model berbasis time series.\newline Kekurangan: Fokus hanya pada data LQ45 tanpa mempertimbangkan pengaruh faktor fundamental saham.\\

    \hline
		\multirow[t]{6}{*}{4.}             & Nama Peneliti                          & \cite{sofi2021perbandingan} \\
		\cline{2-3}
		                                   & Judul Penelitian                       & Perbandingan Algoritma Linear Regression, LSTM, dan GRU dalam Memprediksi Harga Saham dengan Model Time Series \\
		\cline{2-3}
		                                   & Permasalahan dan Tujuan                & Penelitian ini bertujuan untuk membandingkan performa algoritma Linear Regression, LSTM, dan GRU dalam memprediksi harga saham menggunakan data time series. \\
		\cline{2-3}
		                                   & Metode                                 & Dataset harga saham historis diproses untuk melatih model Linear Regression, LSTM, dan GRU. Kinerja model dibandingkan menggunakan metrik RMSE dan R-squared. \\
		\cline{2-3}
		                                   & Hasil                                  & LSTM dan GRU menunjukkan performa yang lebih unggul dibandingkan Linear Regression, dengan LSTM memberikan akurasi terbaik dalam menangkap pola non-linear. \\
		\cline{2-3}
		                                   & Kelebihan dan Kekurangan               & Kelebihan: Memberikan analisis komparatif yang kuat antara algoritma time series populer.\newline Kekurangan: Tidak mengeksplorasi dampak hyperparameter tuning pada performa model.\\

    \hline
		\multirow[t]{6}{*}{5.}             & Nama Peneliti                          & \cite{chairurrachman2022penerapan} \\
		\cline{2-3}
		                                   & Judul Penelitian                       & Penerapan Long Short-Term Memory pada Data Time Series untuk Prediksi Harga Saham PT Indofood CBP Sukses Makmur Tbk (ICBP) \\
		\cline{2-3}
		                                   & Permasalahan dan Tujuan                & Penelitian ini bertujuan mengevaluasi efektivitas LSTM dalam memprediksi harga saham perusahaan tertentu, dalam hal ini PT ICBP. Fokus utama adalah menangkap pola data historis saham yang kompleks. \\
		\cline{2-3}
		                                   & Metode                                 & Data historis saham PT ICBP diproses dan dimasukkan ke dalam model LSTM yang telah dioptimalkan melalui pengaturan hyperparameter. Evaluasi dilakukan dengan membandingkan akurasi prediksi dengan data aktual. \\
		\cline{2-3}
		                                   & Hasil                                  & LSTM berhasil menangkap pola harga saham PT ICBP dengan akurasi yang cukup tinggi. \\
		\cline{2-3}
		                                   & Kelebihan dan Kekurangan               & Kelebihan: Fokus pada perusahaan spesifik, memungkinkan analisis yang mendalam dan terarah.\newline Kekurangan: Tidak memperhatikan pengaruh faktor eksternal seperti kondisi pasar dan ekonomi global. \\

    \hline
    \multirow[t]{6}{*}{6.}             & Nama Peneliti                          & \cite{julian2021peramalan} \\
		\cline{2-3}
		                                   & Judul Penelitian                       & Peramalan Harga Saham Pertambangan pada Bursa Efek Indonesia Menggunakan Long Short Term Memory (LSTM). \\
		\cline{2-3}
		                                   & Permasalahan dan Tujuan                & Penelitian ini membahas penerapan LSTM untuk memprediksi harga saham sektor pertambangan di Bursa Efek Indonesia (BEI). Tujuannya adalah mengevaluasi akurasi prediksi menggunakan LSTM dibandingkan pendekatan lainnya. \\
		\cline{2-3}
		                                   & Metode                                 & Data historis saham sektor pertambangan diolah melalui tahapan preprocessing. Model LSTM dibangun untuk prediksi, dan performanya dibandingkan dengan metode lain menggunakan metrik akurasi seperti RMSE. \\
		\cline{2-3}
		                                   & Hasil                                  & LSTM memberikan akurasi terbaik dibandingkan metode tradisional dalam memprediksi saham sektor pertambangan. \\
		\cline{2-3}
		                                   & Kelebihan dan Kekurangan               & Kelebihan: Memanfaatkan algoritma canggih untuk sektor industri spesifik.\newline Kekurangan: Terbatas pada sektor pertambangan, sehingga kurang relevan untuk sektor lain. \\
	\end{longtable}
\end{center}
\addlongtabletolist{Studi Literatur}
