\textbf{\section{Latar Belakang}}
\indent

Pasar saham merupakan salah satu instrumen investasi yang sangat menarik bagi para investor dikarenakan keuntungan saham yang jauh lebih besar di banding instrumen investasi lain nya (\cite{Shen2020}). Namun saham juga memiliki risiko yang sangatlah tinggi di banding instrumen investasi lainnya. Fluktuasi harga saham yang sering terjadi di pasar saham membuat investor perlu strategi dan keputusan yang tepat untuk meminimalisir risiko dan memaksimalkan keuntungan. Prediksi harga saham yang akurat memiliki dampak yang sangat besar untuk investor dalam membuat keputusan investasi yang tepat. Dengan melihat serta memahami pola-pola pergerakan harga saham terdahulu untuk memprediksi harga saham yang akan terjadi di masa depan.

Saham adalah sebuah dokumen berharga yang mampu menampilkan bagian kepemilikan dari suatu perusahaan yang berarti ketika seseorang memutuskan untuk membeli sebuah saham maka artinya orang tersebut sudah membeli sebagian dari kepemilikan perusahaan yang dibelinya. Dalam arti lain saham juga dapat diartikan sebagai suatu satuan nilai ataupun pembukuan dalam komponen finansial yang berfokus pada bagian bentuk kepemilikan suatu perusahaan. Atau secara sederhana, saham merupakan suatu alat bukti atas kepemilikan pada suatu perusahaan yang biasanya berupa lembaran kertas yang mana isinya menyatakan kepemilikan perusahaan tersebut dari perusahaan yang membuat surat (\cite{soebiantoro2021perdagangan}).

Investasi nyata adalah bentuk investasi yang melibatkan aset berwujud, seperti tanah; gedung; mesin-mesin dan pabrik; sedangkan investasi keuangan (financial investment) merupakan bentuk investasi yang melibatkan kontrak tertulis seperti saham biasa \textit{(common stock)} dan obligasi \textit{(bond)} (\cite{wahyuliantini2015pengaruh}).

Dalam melakukan investasi, untuk meminimalisir risiko dalam memilih saham, ada dua metode analisis yang dapat digunakan, yaitu analisis fundamental dan juga analisis teknikal. Kedua metode tersebut memiliki pendekatan yang berbeda namun saling melengkapi dalam membantu investor dalam membuat keputusan yang lebih tepat.

Analisis fundamental merupakan sebuah metode analisis saham yang berfokus pada faktor-faktor internal dan juga eksternal yang tentunya memengaruhi potensi pertumbuhan perusahaan. Faktor-faktor yang dianalisis meliputi laporan keuangan, manajemen perusahaan, kondisi perusahaan, dan faktor ekonomi yang jauh lebih luas lagi. Dengan melakukan analisis ini investor dapat memahami kondisi keuangan perusahaan serta prospek masa depan sebuah perusahaan dan dapat membantu investor untuk menilai nilai intrinsik saham dan menentukan apakah saham tersebut \textit{undervalued} (terlalu murah) atau \textit{overvalued} (terlalu mahal).

Berbeda dengan analisis fundamental, analisis teknikal ini berfokus pada pola pergerakan harga saham di pasar berdasarkan data historis. Dengan adanya pola grafik serta indikator teknikal seperti \textit{Moving Averages, Relative Strength Index (RSI), dan Bollinger Bands}, analisis teknikal berusaha untuk melakukan identifikasi tren pada pasar dan pola-pola yang dapat menunjukkan kemungkinan pergerakan harga saham di masa depan.

Namun, kedua analisis saham tersebut memiliki keterbatasan. Analisis fundamental memerlukan waktu serta keahlian yang mendalam untuk menilai tingkat kesehatan dari sebuah perusahaan secara menyeluruh, sedangkan analisis teknikal dapat menjadi subjektif karena interpretasi pola grafik yang berbeda-beda. Oleh karena itu, kemajuan teknologi di bidang kecerdasan buatan menawarkan solusi yang inovatif untuk menyelesaikan permasalahan yang ada dari kedua teknik analisis tersebut yang tentunya membantu investor dalam mengambil keputusan dalam proses analisis pasar saham.

% TODO: ganti referensinya
Dalam beberapa dekade terakhir ini perkembangan teknologi terutama pada bidang kecerdasan buatan telah berkontribusi di berbagai bidang salah satunya yaitu di sektor keuangan dan investasi. Berkembangnya teknologi telah mempermudah manusia untuk pengambilan keputusan. Maka dari itu penelitian ini dibuat untuk memudahkan investor dalam mengambil keputusan investasi dengan menggunakan salah satu metode kecerdasan buatan yaitu \textit{deep learning}. Dengan adanya analisis data menggunakan \textit{deep learning} maka investor dapat dengan mudah melihat pola-pola pergerakan harga saham tanpa harus melakukan analisis teknikal mendalam untuk memahami dan memprediksi harga saham (\cite{chairurrachman2022penerapan}).

Deep Learning merupakan sebuah konsep kecerdasan artifisial yang menggunakan jaringan saraf untuk memahami dan mempelajari pola-pola yang kompleks dalam data dengan mengekstrak fitur dari sebuah data (\cite{pipin2023prediksi}). Salah satu algoritma analisis yang terkenal sangat efektif adalah \textit{long short-term memory} (LSTM). Algoritma ini mampu menangkap pola temporal pada data \textit{time series}, sehingga cocok untuk memprediksi harga saham yang cenderung bersifat dinamis dan non-linear (\cite{chairurrachman2022penerapan}). Selain itu keunggulan dari algoritma \textit{LSTM} adalah kemampuan untuk mengatasi masalah \textit{vanishing gradient} yang sering terjadi pada proses pelatihan jaringan saraf \textit{(neural network)} yang dimana \textit{gradien} yang dihitung dalam proses \textit{backpropagation} dapat mendekati angka nol yang berarti algoritma ini sangatlah efisien untuk melakukan prediksi data yang berhubungan dengan data historis (\cite{alim2023pemodelan}). Dengan membaca pola pergerakan harga saham dalam data \textit{time series} yang dikumpulkan, LSTM dapat mempelajari hubungan temporal dalam data tersebut yang digunakan untuk memprediksi harga saham yang akan terjadi di masa depan. LSTM mampu menangani masalah perhitungan urutan dan waktu yang panjang, yang sering kali terjadi pada data keuangan seperti pergerakan harga saham. LSTM memberikan keuntungan besar dalam analisis teknikal saham, yang dapat membantu investor dalam pengambilan keputusan investasi berdasarkan prediksi yang lebih terpercaya (\cite{sofi2021perbandingan}).

Penelitian ini akan menggunakan data historis yang akan dikumpulkan menggunakan emiten atau perusahaan Indonesia yang terdaftar di LQ45 dan terdapat pada \textit{Yahoo Finance}, yang juga terdaftar di Bursa Efek Indonesia (BEI) (\cite{julian2021peramalan}). Data ini akan dikumpulkan dan diolah menggunakan \textit{Python} dan \textit{Pandas}. Data yang dikumpulkan akan digunakan dalam algoritma \textit{LSTM} untuk memprediksi harga saham di Bursa Efek Indonesia.

Berbeda dengan penelitian-penelitian sebelumnya, penelitian ini berfokus pada penerapan sistem pendukung keputusan investasi yang mencakup hampir seluruh saham di Indonesia, khususnya saham dari LQ45. Selain itu, penelitian sebelumnya yang dilakukan oleh peneliti seperti Pipin, Sio Jurnalis, dan Purba (2003) menggunakan algoritma \textit{Recurrent Neural Network} (RNN), yang dalam penelitian terbaru menunjukkan bahwa LSTM memiliki akurasi yang lebih unggul dibandingkan dengan RNN (\cite{alim2023pemodelan}). Keunggulan RNN terletak pada kecepatan proses \textit{training} data, namun LSTM lebih efisien dan akurat dalam memprediksi data historis. Oleh karena itu, penelitian ini mengutamakan penggunaan algoritma LSTM yang memiliki akurasi lebih tinggi dibandingkan dengan algoritma lainnya.

Maka dari itu penelitian ini memiliki fokus dengan algoritma LSTM dikarenakan tingkat akurasi yang unggul jika dibandingkan dengan algoritma yang lainnya. Dalam penelitian ini, akan dilakukan analisis data saham Indonesia serta menguji kemampuan LSTM dalam memprediksi pergerakan harga saham secara lebih akurat. Penelitian ini juga akan mengembangkan sistem yang lebih mudah untuk digunakan (\textit{user-friendly}) dan juga aplikatif yang memungkinkan dapat digunakan oleh investor yang mungkin minim terhadap analisis teknikal mendalam. Dengan demikian, penelitian ini menawarkan pendekatan yang lebih praktis serta memberikan kontribusi baru terhadap literatur yang ada terkait analisis saham di Indonesia.
