\textbf{\section{LATAR BELAKANG}}
\indent

Pasar saham merupakan salah satu instrumen investasi yang sangat menarik bagi para investor dikarenakan keuntungan saham yang jauh lebih besar dibanding instrumen investasi lainnya (\cite{Shen2020}). Namun saham juga memiliki resiko yang sangatlah tinggi dibanding instrumen investasi lainnya. Fluktuasi harga saham yang sering terjadi di pasar saham membuat investor perlu strategi dan keputusan yang tepat untuk meminimalisir resiko adn memaksimalkan keuntungan. Dengan adanya teknologi analisis data, maka dapat dilakukan prediksi berdasarkan data historical. Salah satu algoritma yang cocok untuk mengolah data historical adalah \textit{LONG SHORT-TERM MEMORY} (LSTM). \textit{LSTM} merupakan sebuah algoritma pembelajaran mendalam (deep learning) yang mampu menangkap pola temporal dalam data historical. Dengan analisis ini dapat memudahkan investor dalam pengambilan keputusan investasi.

Dalam beberapa dekade terkahir ini perkembangan teknologi terutama pada bidang kecerdasan buatan. Kecerdasan buatan telah berkontribusi besar pada sektor analisis dan pengolahan data keuangan. Salah satu pendekatan yang terkenal sangat efektif adalah \textit{LONG SHORT-TERM MEMORY} (LSTM). Algoritma ini mampu menangkap pola temporal pada data time series, sehingga cocok untuk memprediksi harga saham yang cenderung bersifat dinamis dan non-linear (\cite{chairurrachman2022penerapan}).

Penelitian ini akan menggunakan data historis yang akan dikumpulkan menggunakan emiten atau perusahaan Indonesia yang terdapat pada \textit{Yahoo Finance} yang tentu saja terdaftar juga pada Bursa Efek Indonesia (BEI) (\cite{julian2021peramalan}). Data ini akan dikumpulkan menggunakan \textit{Python} dan \textit{Pandas} untuk mengolah data. Data yang dikumpulkan akan dikumpulkan menggunakan algoritma \textit{LSTM} yang akan digunakan untuk memprediksi harga saham pada bursa efek Indonesia. Data yang dikumpulkan akan dikumpulkan menggunakan algoritma \textit{LSTM} yang akan digunakan untuk memprediksi harga saham pada bursa efek Indonesia.

\pagebreak
